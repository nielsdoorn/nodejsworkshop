\documentclass[a4paper]{report}

\usepackage{NielsPackage}

\lstset{language=JavaScript}

\hypersetup{
	pdfauthor = {Niels Doorn},
	pdftitle = {Node.js Twitter Lab},
	pdfsubject = {Node.js, workshop, JavaScript},
	pdfkeywords = {Node.js},
	pdfcreator = {NielsDoorn/EnschedeWebDevelopersMeetup/RocVanTwente}
}

\rhead{Node.js Twitter Lab}
\lhead{Observe Hack Make}
\chead{}
\lfoot{Niels Doorn \copyright~2013}
\cfoot{}
\rfoot{\thepage}

\fancypagestyle{plain}{
	\fancyhf{}
	\fancyfoot[L]{Niels Doorn \copyright~2013}
	\fancyfoot[C]{}
	\fancyfoot[R]{\thepage}
	\renewcommand{\headrulewidth}{0pt}
	\renewcommand{\footrulewidth}{0.4pt}
}

\begin{document}

\chapter*{\textcolor{seccol}{Node.js} Twitter Lab}

\section*{Introduction}
In this lab we'll make a simple node app that retrieves tweets containing a keyword. It runs in a terminal, but you can easily turn it into a web app.

You'll need a twitter account for this lab and a api key.

\subsection*{Step 1: setup and dependencies}

\begin{lstlisting}[language=bash]
touch package.json
touch twitter.js
\end{lstlisting}

\noindent We'll use the \texttt{ntwitter} module to connect with twitter, therefore we add a dependency in the \texttt{package.json} file:
\lstinputlisting[language=JavaScript]{code/lab/package.json.1}

After that we can install the dependencies by running npm:
\begin{lstlisting}[language=bash]
npm install
\end{lstlisting}

\subsection*{Step 2: Client API keys}

\noindent We need some client keys from twitter for this. You can get them at \url{https://dev.twitter.com}. We set them as environment variables to avoid having o hardcode them.
\begin{lstlisting}[language=bash]
export TWITTER_CONSUMER_KEY=xxxxxxxxxxxxx
export TWITTER_CONSUMER_SECRET=xxxxxxxxxxxxx
export TWITTER_ACCESS_TOKEN_KEY=xxxxxxxxxxxxx
export TWITTER_TOKEN_SECRET=xxxxxxxxxxxxx
\end{lstlisting}

\subsection*{Step 3: using the API}

\noindent Now we can use the API, let's try a simple search in the terminal:
\lstinputlisting[language=JavaScript]{code/lab/twitter-terminal.js}

Now we can run it ans see what happens:
\begin{lstlisting}[language=bash]
node twitter.js
\end{lstlisting}

\subsection*{Step 4: create a webserver}
Ok, thats all very nice, but it would be much more fun if we could see all those incredible tweets on a website. So let's create a webserver that updates the page every time a new tweet comes in.

If we want to create a nice website for this, we'll probably end up serving some static content like css files, images and javascript files for the client. We are going to use \texttt{node-static} for that. To install that module we'll update the \texttt{package.json} file to include this new dependency.

\lstinputlisting[language=JavaScript]{code/lab/package.json}

After that we can run npm again to install it:
\begin{lstlisting}[language=bash]
npm install
\end{lstlisting}

\noindent Now we create the webserver in our twitter app by adding these lines:

\begin{lstlisting}[language=JavaScript]
var http = require('http');
var static = require('node-static');

// http server
var app = http.createServer(handler);
app.listen(1337);
var file = new static.Server('./public');
function handler(req, res) {
  file.serve(req, res);
}
\end{lstlisting}

\noindent We've configured the webserver to serve static content from the public folder, so we'll have to create that folder:
\begin{lstlisting}[language=bash]
mkdir public
touch pubic/index.html
touch public/client.js
\end{lstlisting}

\noindent in that folder we can put all our static content like a HTML file:

\lstinputlisting[language=HTML]{code/lab/public/index.html}

\subsection*{Step 5: Tweet events}

\noindent and client side javascript:

\lstinputlisting[language=JavaScript]{code/lab/public/client.js}

\end{document}